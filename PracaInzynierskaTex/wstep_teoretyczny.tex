\chapter{Wstęp teoretyczny}
\label{cha:wstepTeoretyczny}

%---------------------------------------------------------------------------

\section{Procesy biznesowe}
\label{sec:procesyBiznesowe}
\subsection{Procesy Biznesowe}
\subsection{Dzienniki zdarzeń}

%---------------------------------------------------------------------------

\section{Modelowanie procesów biznesowych}
\label{sec:modelowanie}

%---------------------------------------------------------------------------

\section{Algorytmy do wykrywania procesów biznesowych}
\label{sec:algorytmy}
\subsubsection{Alpha algorithm}
\subsubsection{The ILP Miner}
\subsubsection{Heuristic Miner}
\subsubsection{Multi-phase Miner}

%---------------------------------------------------------------------------

\section{Ewolucja genetyczna}
\label{sec:ewolucjaGenetyczne}
\subsection{Algorytmy genetyczne}
Pierwszy raz zaproponowane w \cite{10.5555/138936}
\subsection{Ewolucja genetyczna a inne algorytmy uczenia maszynowego}
\subsection{Ewolucja gramatyczna}

%---------------------------------------------------------------------------

\section{Gramatyka}
\label{sec:gramatyka}
\subsection{BNF}
\subsection{Tworzenie gramatyki pod kątem ewolucji}

Tworząc gramatykę pod kątem wykorzystania jej w procesie ewolucji ważne jest, żeby ilość produkcji jak najlepiej odzwierciedlała jak często chcemy uzyskać dany stan.

%---------------------------------------------------------------------------

\section{Metryki}
\label{sec:metryki}
\cite{doi:10.1142/S0218843014400012}
\subsection{Prostota}
$M_{pro} = 1 - \frac{ilosc\ duplikatow\ w\ modelu\ +\ ilosc\ brakujacych\ wartosci\ w\ modelu}{ilosc\ unikalnych\ zdarzen\ w\ logu\ +\ ilosc\ zdarzen\ w\ modelu}$
\subsection{Odwzorowanie}
$M_o = (1 - \sum_{0}^{ilosc\ procesow\ w\ logu} \frac{blad\ odwzorowania\ logu\ w\ modelu}{minimalna\ długosc\ sciezki\ w\ modelu\ +\ długosc\ sciezki\ w\ logu})^4$
\subsection{Precyzja}
$M_{pre} = (1 - \sum_{0}^{ilosc\ zdarzen\ w\ modelu} \frac{ilosc\ osiagalnych\ zdarzen\ w\ modelu - ilosc\ osiagalnych\ zdarzen\ w\ logu}{ilosc\ osiagalnych\ zdarzen\ w\ modelu})^{\frac{1}{3}} $
\subsection{Generalizacja}
$M_g = \frac{1 - \sum_{0}^{ilosc\ zdarzen\ w\ logu} \frac{1}{\sqrt{ilosc\ wystapien\ zdarzenia}}}{ilosc\ zdarzen\ w\ logu} $
\subsection{Złożoność}
$M_z = 1 - \frac{1}{\sqrt{1 - odwzorowanie\ *\ \sqrt{zlozonosc\ modelu}}} $
