\chapter{Wprowadzenie}
\label{cha:wprowadzenie}

%---------------------------------------------------------------------------

\section{Zarys tematyki pracy}
\label{sec:zarysPracy}
W związku z możliwością gromadzenia coraz większej ilości danych, a także chęcią ich wykorzystania rosnącą popularnością Data Science 
Procesy biznesowe mogą pozwolić na przewidywanie przyszłych zdarzeń na podstawie danych, znajdowanie wąskich gardeł,

Analiza procesów biznesowych może pozwolić na zwiększenie produktywności oraz redukcję kosztów
 


\section{Cele pracy}
\label{sec:celePracy}

Celem pracy jest projekt i implementacja metody odkrywania procesów biznesowych przy użyciu programowania genetycznego. W pracy zbadano jak wybór metod programowania genetycznego, wybór gramatyki, a także parametrów programu wpływa na jakość rozwiązania. Zaprezentowano też przykłady użycia algorytmu do okrywania procesów biznesowych oraz porównano z innymi dostępnymi algorytmami. Ponadto w pracy zostały zbadana hipoteza czy rozwiązywania problemu najpierw dla prostych przypadków i wykorzystanie rozwiązań tego problemu może mieć korzystny wpływ na rozwiązanie bardziej skomplikowanego problemu.
%---------------------------------------------------------------------------

\section{Zawartość pracy}
\label{sec:zawartoscPracy}

Praca zastała podzielona na cztery części. We wstępie teoretycznym zostały przybliżone zagadnienia potrzebne do zrozumienia pracy. W kolejnej części została przedstawiona implementacja algorytmu do wyszukiwania procesów genetycznych. Następnie zaprezentowane zostały wyniki działanie algorytmu dla przykładowych dzienników zdarzeń. Omówione zostało też jak na czas znajdowania rozwiązania oraz jego jakość wpływają przyjęte parametry algorytmu w szczególności wybór metryk oraz wagi z jakimi każda metryka powinna być brana pod uwagę.  


















