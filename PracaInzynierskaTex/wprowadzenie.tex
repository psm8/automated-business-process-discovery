\chapter{Wprowadzenie}
\label{cha:wprowadzenie}

%---------------------------------------------------------------------------

\section{Zarys tematyki pracy}
\label{sec:zarysPracy}

Zdefiniowanie kroków potrzebnych do osiągnięcia danego efektu jest konieczne do zrozumienia podejmowanych działań i wprowadzenia ewentualnych udoskonaleń. Z czasem biznes zdał sobie z tego sprawę i kierując się zasadą: ,,Jeżeli nie jesteś w stanie opisać czegoś jako proces, nie masz pojęcia, co robisz'', firmy zaczęły podejmować próby uporządkowania i zamknięcia swoich działań w ramy, co doprowadziło do wzrostu popularności procesów biznesowych.

Identyfikacja i opis procesów biznesowy sprawia, że wszystkie operacje w firmie stają się przejrzyste i łatwiejsze do zrozumienia. Analiza procesów biznesowych może pozwolić na zwiększenie produktywności oraz redukcję kosztów. Procesy biznesowe mogą pozwolić na przewidywanie przyszłych zdarzeń na podstawie danych, znajdowanie wąskich gardeł, a także zmniejszają zależność firm od poszczególnych ludzi.

W związku z możliwością gromadzenia coraz większej ilości danych, a także chęcią ich wykorzystania oraz rosnącą popularnością analizy danych (\textit{eng. data science}), biznes zdał sobie sprawę z możliwości wykorzystanie technologii informatycznych w kontekście procesów biznesowych. Zapoczątkowało to powstanie na pograniczu zarządzania procesami biznesowymi i metod informatycznych używanych do analizy danych, wśród wielu innych, dziedziny zwanej eksploracją procesów (\textit{eng. process mining}).
 
\section{Cele pracy}
\label{sec:celePracy}

Celem pracy jest projekt i implementacja metody odkrywania procesów biznesowych przy użyciu programowania genetycznego, a dokładniej ewolucji gramatycznej. W ramach pracy zaimplementowano program realizujący to zadanie oraz zbadano jak wybór metryk, metod ewolucji, gramatyki, a także parametrów programu wpływa na jakość rozwiązania. Działanie zweryfikowano poprzez użycie stworzonego algorytm do okrycia modeli procesów biznesowych dla dzienników zdarzeń różnej wielkości. Przykłady czego zamieszczono i omówiono w pracy. Ponadto w pracy została zbadana hipoteza, czy kontrola złożoności modelu na etapie ewolucji ma korzystny wpływ na działanie algorytmu i ostateczne rozwiązanie.
%---------------------------------------------------------------------------

\section{Zawartość pracy}
\label{sec:zawartoscPracy}

Praca zastała podzielona na cztery części. We wstępie teoretycznym zostały przybliżone zagadnienia potrzebne do zrozumienia pracy, takie jak procesy biznesowe, eksploracja procesów oraz ewolucja gramatyczna. Omówiono też wybór i działanie metryk dla modeli procesów biznesowych. W kolejnej części została przedstawiona gramatyka stworzona na potrzebny odkrywania procesów oraz projekt i implementacja algorytmu do wyszukiwania procesów genetycznych. Następnie zaprezentowane zostały wyniki działanie algorytmu dla przykładowych dzienników zdarzeń. Sprawdzone zostało jak na czas znajdowania rozwiązania oraz jego jakość wpływają przyjęte parametry algorytmu w szczególności wybór metryk, oraz wagi, z jakimi każda metryka powinna być brana pod uwagę. Zweryfikowana została hipoteza dotycząca złożoności modelu. Na koniec przedstawiono podsumowanie całości pracy. 


















