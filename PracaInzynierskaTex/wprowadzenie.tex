\chapter{Wprowadzenie}
\label{cha:wprowadzenie}

%---------------------------------------------------------------------------

\section{Zarys tematyki pracy}
\label{sec:zarysPracy}

Zdefiniowanie kroków potrzebnych do osiągnięcia danego efektu jest konieczne do zrozumienia podejmowanych działań i wprowadzenia ewentualnych udoskonaleń. Z czasem biznes zdał sobie z tego sprawę i kierując się zasadą: ,,Jeżeli nie jesteś w stanie opisać czegoś jako proces, nie masz pojęcia, co robisz'', firmy zaczęły podejmować próby uporządkowania i zamknięcia swoich działań w ramy, co doprowadziło do wzrostu popularności procesów biznesowych.

Identyfikacja i opis procesów biznesowy sprawia, że wszystkie operacje w firmie stają się przejrzyste i łatwiejsze do zrozumienia. Analiza procesów biznesowych może pozwolić na zwiększenie produktywności oraz redukcję kosztów. Procesy biznesowe mogą pozwolić na przewidywanie przyszłych zdarzeń na podstawie danych, znajdowanie wąskich gardeł, a także zmniejszają zależność firm od poszczególnych ludzi.

W związku z możliwością gromadzenia coraz większej ilości danych, a także chęcią ich wykorzystania oraz rosnącą popularnością analizy danych (\textit{eng. data science}), biznes zdał sobie sprawę z możliwości wykorzystanie technologii informatycznych w kontekście procesów biznesowych. Zapoczątkowało to powstanie na pograniczu zarządzania procesami biznesowymi i metod informatycznych używanych do analizy danych, wśród wielu innych, dziedziny zwanej eksploracją procesów (\textit{eng. process mining}).
 
\section{Cele pracy}
\label{sec:celePracy}

Celem pracy jest projekt i implementacja metody odkrywania procesów biznesowych przy użyciu programowania genetycznego. W pracy zbadano jak wybór metod programowania genetycznego, wybór gramatyki, a także parametrów programu wpływa na jakość rozwiązania. Zaprezentowano też przykłady użycia algorytmu do okrywania procesów biznesowych oraz porównano z innymi dostępnymi algorytmami. Ponadto w pracy zostały zbadana hipoteza czy rozwiązywania problemu najpierw dla prostych przypadków i wykorzystanie rozwiązań tego problemu może mieć korzystny wpływ na rozwiązanie bardziej skomplikowanego problemu.
%---------------------------------------------------------------------------

\section{Zawartość pracy}
\label{sec:zawartoscPracy}

Praca zastała podzielona na cztery części. We wstępie teoretycznym zostały przybliżone zagadnienia potrzebne do zrozumienia pracy, takie jak procesy biznesowe, eksploracja procesów oraz ewolucja gramatyczna. W kolejnej części została przedstawiona gramatyka stworzona na potrzebny odkrywania procesów oraz projekt i implementacja algorytmu do wyszukiwania procesów genetycznych. Następnie zaprezentowane zostały wyniki działanie algorytmu dla przykładowych dzienników zdarzeń. Omówione zostało też jak na czas znajdowania rozwiązania oraz jego jakość wpływają przyjęte parametry algorytmu w szczególności wybór metryk oraz wagi z jakimi każda metryka powinna być brana pod uwagę.  


















