\chapter{Projekt i implementacja}

\section{Wykorzystane technologie}
\subsection{Python 3.8.1}
\subsection{PonyGE2}



\section{Tworzenie gramatyki procesu biznesowego}

“a token is a theoretical concept that is used as an aid to define the behaviour of a process that is being
performed”

\section{Implementacja}

Ogólny flow:


Parsowanie gramatyki:
Wyszukiwanie w modelu logów o określonej długości:
calculate alignment:
traceback:
calculate other metrics:

\begin{figure}[!ht]
\lstset{caption=Parser gramatyki, captionpos=b}
\lstset{label=src:passive, frame=single}
\begin{lstlisting}
agent Buffer {
  i :: Int = 0;
  proc pop  { out pop i; }
  proc push { in  push i; }
}
\end{lstlisting}
\end{figure}

\section{Wybór parametrów algorytmu}
